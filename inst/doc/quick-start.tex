\documentclass[english]{article}
\usepackage{/usr/lib/R/share/texmf/Sweave}
\begin{document}

\title{Quick introduction to the rsbml package}
%\VignetteIndexEntry{Quick start for rsbml}
\author{Michael Lawrence}

\maketitle

\section*{Importing an SBML document}
Most users will begin an rsbml session by importing an SBML file into R.
In the example below, we load an SBML file describing the glycolysis pathway.
It is also possible to parse an R character vector instead of an external file.

\begin{Schunk}
\begin{Sinput}
> library(rsbml)
> doc <- rsbml_read(system.file("sbml", "GlycolysisLayout.xml", 
+     package = "rsbml"))
\end{Sinput}
\end{Schunk}

If errors are encountered, the function throws an error along with
warnings describing the specific problem(s) with the document.
Otherwise, the result is an opaque object referring to a low-level 
libsbml data structure. From here, the user currently has two options for 
accessing the data: as an S4 object conforming to the SBML document object 
model or as a Bioconductor graph object.

\section*{The S4 object representation}
Converting the opaque libsbml parse result to an S4 object is simple:

\begin{Schunk}
\begin{Sinput}
> dom <- rsbml_dom(doc)
\end{Sinput}
\end{Schunk}

The result contains all of the information from the SBML. Methods exist for
accessing every element of the SBML specification (up to L2V1). Here is how
one would retrieve all of the species ID's from the SBML model:

\begin{Schunk}
\begin{Sinput}
> sapply(species(model(dom)), id)
\end{Sinput}
\begin{Soutput}
 [1] "Glucose"                   "Glucose-6-phosphate"      
 [3] "Fructose-6-phosphate"      "Fructose-1_6-bisphosphate"
 [5] "Dihydroxyacetonephosphate" "Glyceraldehyd-3-phosphate"
 [7] "1_3-Bisphosphoglycerate"   "3-Phosphoglycerate"       
 [9] "2-Phosphoglycerate"        "Phosphoenolpyruvate"      
[11] "Pyruvate"                  "ATP"                      
[13] "ADP"                       "H+"                       
[15] "NAD+"                      "NADH"                     
[17] "H2O"                       "Pi"                       
\end{Soutput}
\end{Schunk}

The user may also modify every part of the model. Note that very little 
validation is performed in response to modifications. See \ref{sec:checking} for
a guide to checking SBML models for consistency problems.

\section*{The graph representation}
All SBML models have an implicit graphical structure. To extract this into
a Bioconductor graph object, type the following:

\begin{Schunk}
\begin{Sinput}
> g <- rsbml_graph(doc)
> nodes(g)
\end{Sinput}
\begin{Soutput}
 [1] "Glucose"                    "Glucose-6-phosphate"       
 [3] "Fructose-6-phosphate"       "Fructose-1_6-bisphosphate" 
 [5] "Dihydroxyacetonephosphate"  "Glyceraldehyd-3-phosphate" 
 [7] "1_3-Bisphosphoglycerate"    "3-Phosphoglycerate"        
 [9] "2-Phosphoglycerate"         "Phosphoenolpyruvate"       
[11] "Pyruvate"                   "ATP"                       
[13] "ADP"                        "H+"                        
[15] "NAD+"                       "NADH"                      
[17] "H2O"                        "Pi"                        
[19] "Hexokinase"                 "Phosphoglucoseisomerase"   
[21] "Phosphofructokinase"        "Aldolase"                  
[23] "triose_phosphate_isomerase" "GAP_Dehydrogenase"         
[25] "Phosphoglyceratekinase"     "Phosphoglyceratemutase"    
[27] "Enolase"                    "Pyruvatekinase"            
\end{Soutput}
\end{Schunk}

Note that the list of node ID's contains all of the species ID's retrieved from
the S4 object above. The additional ID's are for the reaction nodes.
At this point, the graph can be passed to other packages, such as RBGL, Rgraphviz, etc.

\section*{Checking SBML models for consistency}\label{checking}
The SBML specification provides many complex rules that ensure an SBML model
is internally consistent. The following is an example of checking a document
against those rules.

\begin{Schunk}
\begin{Sinput}
> rsbml_check(doc)
\end{Sinput}
\begin{Soutput}
[1] TRUE
\end{Soutput}
\end{Schunk}

If there were any problems with the document, they would be communicated as
warnings. If you would like to compute on the problems, the following returns
a data structure describing the errors along with their positions in the SBML,
if available.

\begin{Schunk}
\begin{Sinput}
> doc <- rsbml_read("non-existent-file.xml", check = FALSE)
> rsbml_problems(doc)
\end{Sinput}
\begin{Soutput}
$warnings
list()

$errors
list()

$fatals
$fatals[[1]]
$fatals[[1]]$line
[1] 0

$fatals[[1]]$column
[1] 0

$fatals[[1]]$message
[1] "File 'non-existent-file.xml' does not exist."
\end{Soutput}
\end{Schunk}

\section*{Writing SBML documents}
After creating/manipulating SBML objects in R, the result may be translated
back to XML in two different ways: directly to a file or to an R character vector.
This example is of the latter:

\begin{Schunk}
\begin{Sinput}
> xml <- rsbml_xml(doc)
\end{Sinput}
\end{Schunk}

The result is a string XML representation of the SBML model.

\section*{More information}
For more details, please see the online help for the rsbml package. If you
encounter problems, please email lawremi at the domain iastate.edu.

\end{document}
